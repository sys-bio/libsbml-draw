%%%%%%%%%%%%%%%%%%%%%%%%%%%%%%%%%%%%%%%%
% Class options                        %
%%%%%%%%%%%%%%%%%%%%%%%%%%%%%%%%%%%%%%%%
% Orientation:                         %
% portrait (default), landscape        %
%                                      %
% Paper size:                          %
% a0paper (default), a1paper, a2paper, %
% a3paper, a4paper, a5paper, a6paper   %
%                                      %
% Language:                            %
% english (default), norsk             %
%%%%%%%%%%%%%%%%%%%%%%%%%%%%%%%%%%%%%%%%
\documentclass{uioposter}


\usepackage[absolute, overlay]{textpos}            % Figure placement
\setlength{\TPHorizModule}{\paperwidth}
\setlength{\TPVertModule}{\paperheight}


\title{Libsbml-draw: A rendering library for SBML Layout and Render Extension}
\author
{%
    Natile Hawkins\inst{1}
    \and
    J. Kyle Medley\inst{1}
    \and
    Kiri Choi\inst{1}
    \and
    Herbert Sauro\inst{1}
}
%% Optional:
\institute
{
    \inst{1} Department of Mathematics
    \and
    \inst{2} Department of Informatics
}
% Or:
%\institute{Contact information}


%% Remove footline:
%\setbeamertemplate{footline}{}


\begin{document}
\begin{frame}
\begin{columns}[onlytextwidth]


\begin{column}{1.0\textwidth - 1.5cm}
    \begin{block}{\center{Introduction}}
        The SBML layout and render extensions enable SBML models to encode information about the graphical
        depiction of model elements. The \lq\lq{Layout} extension provides information about the positions
        of model elements, whilst the \lq\lq{Render} extension describes their physical characterics,
        such as their shape, color, line width, and font attributes. Libsbml\_draw is a Python library that
        supports the SBML Layout and Render extensions by automating the generation of layout information for
        SBML models. This is done using a C/C++ backend library called SBNW (unpublished work).
    \end{block}

%    \begin{exampleblock}{Does it come in black?}
%        Sure, use an \textbf{exampleblock}!
%    \end{exampleblock}
%
%    \begin{alertblock}{How do you make it pop?}
%        Use an \alert{alertblock}!
%    \end{alertblock}
%
%    \begin{block}{Method}
%        \lipsum[1]
%    \end{block}
%
%    \begin{block}{Results}
%        \lipsum[2]
%        \missingfigure{Striking imagery relevant to the research}
%        \unskip
%    \end{block}
\end{column}


%\begin{column}{0.5\textwidth - 1.5cm}
%    \begin{block}{Conclusions}
%        \lipsum[4]
%    \end{block}
%
%    \begin{block}{Acknowledgements}
%        \lipsum[5]
%    \end{block}
%
%    \begin{block}{References}
%        \lipsum[6]
%    \end{block}
%
%    \begin{block}{Contact information}
%        \lipsum[75]
%    \end{block}
%\end{column}


\end{columns}


\begin{textblock}{0.5}(0.18, 0.94)
    \color{white}
    \sffamily
    \textbf{Write here using textblock}
    \\
    Such as contact information or references
\end{textblock}


\end{frame}
\end{document}